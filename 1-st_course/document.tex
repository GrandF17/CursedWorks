\documentclass[utf8,14pt,a4paper,oneside,russian]{book}
\usepackage[14pt]{extsizes}
\usepackage{longtable}

%===========
%=Кодировка=
%===========
\usepackage[T2A]{fontenc}
\usepackage[utf8]{inputenc}
\usepackage[main=russian, english]{babel}

%===================
%=Разметка страницы=
%===================
\usepackage[left=3cm, right=1cm, top=2cm, bottom=2cm, headheight=14pt, headsep=1cm, footskip=1cm]{geometry}
\pagestyle{plain}
\linespread{1.1} %Межстрочный интервад
\setlength{\parindent}{1.25cm} %Абзацный отступ
\setlength{\parskip}{0em} %Интервал между абзацами
\usepackage{indentfirst}

\usepackage{misccorr}
\usepackage{graphicx}
\usepackage{amsmath, amssymb, amsfonts}

%===========
%=Заголовки=
%===========
\usepackage{titlesec}

\titleformat{\section}{\centering\large\bfseries}{\thesection.}{0.5em}{\MakeUppercase}
\renewcommand{\thesection}{\arabic{section}}
\renewcommand{\sectionmark}[1]{\markright{\thesection.~#1}}
\titlespacing*{\section}{0em}{2em}{1em}

\titleformat{\subsection}{\centering\normalsize\bfseries}{\thesubsection.}{0.5em}{}
\renewcommand{\thesubsection}{\arabic{section}.\arabic{subsection}}
\titlespacing*{\subsection}{0em}{1.25em}{0.5em}

\titlespacing*{\paragraph}{0em}{1.05em}{0.25em}

%============
%=Содержание=
%============
\usepackage{titletoc}
\makeatletter
\renewcommand{\tableofcontents}{\section*{Содержание}\markboth{Содержание}{}\@starttoc{toc}\newpage}
\makeatother
\titlecontents{section}[1.5em]{}{\contentslabel[\thecontentslabel.]{1.5em}}{}{\rule{0.1cm}{0pt}\titlerule*[0.75pc]{.}\contentspage}
\titlecontents{subsection}[4em]{\vspace{0.05em}}{\contentslabel[\thecontentslabel.]{2em}}{}{\rule{0.1cm}{0pt}\titlerule*[0.75pc]{.}\contentspage}

%========
%=Списки=
%========
\usepackage{enumitem}
\makeatletter
\AddEnumerateCounter{\asbuk}{\@asbuk}{м)}
\makeatother
\setlist{nolistsep, topsep=0.375em}
\setenumerate{leftmargin=2cm, labelsep=0em, labelwidth=0.75cm, align=left}
\setitemize{leftmargin=2cm, labelsep=0em, labelwidth=0.75cm, align=left}
\renewcommand{\labelitemi}{--}
\renewcommand{\labelenumi}{\arabic{enumi})}
\renewcommand{\labelenumii}{\asbuk{enumii})}
\renewcommand{\labelenumiii}{--}

%==========================
%=Математические операторы=
%==========================
\DeclareMathOperator{\mob}{Mob}
\DeclareMathOperator{\fix}{Fix}
\DeclareMathOperator{\ord}{ord}

\graphicspath{{pictures/}}
\DeclareGraphicsExtensions{.pdf,.png,.jpg}

\begin{document}
	
\thispagestyle{empty}
\small
\begin{center}
%	\includegraphics[width=2.55cm]{m}\\
	\MakeUppercase{Минобрнауки России}\\[1em]
	Федеральное государственное бюджетное образовательное учреждение\\
	высшего образования\\[0.5em]
	\textbf{<<МИРЭА -- Российский технологический университет>>}\\
	\textbf{РТУ МИРЭА}\\
	\rule{\textwidth}{0.75pt}\\
	Институт кибернетики\\
	Базовая кафедра №252 -- информационной безопасности\\[-0.45em]
	\rule{\textwidth}{0.75pt}\\[5em]
	\normalsize\MakeUppercase{\textbf{Курсовая работа}}\small\\[0.5em]
	По дисциплине <<Группы подстановок>>\\[1.5em]
	\textbf{Тема курсовой работы:} <<Порядки классов сопряжённых элементов\\
	в конечных группах подстановок>>(вариант№14)\\[3em]
	\begin{tabular}{p{7cm}p{6cm}c}
		Студент группы ККСО-03-19 & Николенко В.О. & \rule{2cm}{0.75pt}\\[-0.5em]
		& & \footnotesize\textit{(подпись)}\small\\[1em]
		Руководитель курсовой работы & к.ф.-м.н., проф. Зязин В.П. & \rule{2cm}{0.75pt}\\[-0.5em]
		& & \footnotesize\textit{(подпись)}\small\\[1em]
		Консультант & асс. Плешаков А.С. & \rule{2cm}{0.75pt}\\[-0.5em]
		& & \footnotesize\textit{(подпись)}\small\\[5em]
		Работа представлена к защите & <<\rule{0.5cm}{0.75pt}>> \rule{2cm}{0.75pt} 2020 г. & \\[1em]
		Допущен к защите & <<\rule{0.5cm}{0.75pt}>> \rule{2cm}{0.75pt} 2020 г. & \\[1em]
	\end{tabular}
	\vfill
	Москва -- 2020
\end{center}
\normalsize
\newpage
	
	% Содержание
	\tableofcontents
	
	% Первая глава
	\section{Введение}
	В мире на данный момент самый важный ресурс это информация. Не даром Натан Ротшильд сказал:$"$Кто владеет информацией — тот владеет миром$"$. Еще в древние времена, при Юлии Цезаре был придуман один из старейших шифров - $"$Шифр Цезаря$"$. Хоть он был весьма примитивен, но сумел дать толчок сфере защиты информации, ведь даже, если гонца с важным донесением перехватили, но не сумели прочитать, что написано в послании, то планы великих полководцев и не только не будут подвержены риску быть нарушенными людьми со злыми намерениями. Так постепенно и появился предмет "алгебра". Как писал в своей книге М.М. Глухов,$"$термин $"$алгебра$"$ происходит от названия сочинения узбексого математика 9 века Муххамеда ал-Хорезми $"$Альджебр аль-Мукабала$"$, в котором были систематизированны  сведения о правилах действий с числами и общих приёмах решения задач, сводящихся к решению уравнений 1-й и 2-й степеней$"$. Сегодня же $"$алгебра$"$ - это предмет, без которого невозможно работать в сфере защиты информации. Алгебра даёт понять, какие операции можно производить над числами/цифрами и что в итоге у нас получается.
	\newpage
	\section{Теоретическая часть}
	\subsection{Основы теории групп}
	\subsubsection{Задание №9(л)}
	Найти порядок элемента $g$ в группе $G$.
	
	\underline{Решение:}\[ G = C^{*}_{n\times n}g = \left( 
	\begin{array}{cccc}
	\lambda_{1}&0&\cdots&0\\
	0&\lambda_{2}&\cdots&0\\
	\vdots&\vdots&\ddots&\vdots\\
	0&0&\cdots&\lambda_{n}
	\end{array} 
	\right) \]
	, где $\lambda_{1},\lambda_{2},\cdots,\lambda_{n}$ $\in$ Г$_{n}$
	
	Рассмотрим какую-либо конечную группу Г$_{2}$, $\{e^{i\Pi};1\}$ $\in$ Г$_{2}$.
	
	Тогда составим матрицу:
	
	\[ G = \left( 
	\begin{array}{cccc}
	e^{i\Pi}&0\\
	0&1
	\end{array} 
	\right) \]
	
	По определению 1, порядком элемента $g$ группы $G$ называется наименьшее из чисел $n$ $\in$ $N$, при котором $g^{n}=e$, если такие n существуют, и бесконечность - в противном случае.
	
	Результатом перемножения двух матриц является матрица:
	
	$$
	\begin{pmatrix} 
	e^{i\Pi} & 0 \\
	0 & 1
	\end{pmatrix}   \cdot
	\begin{pmatrix}
	e^{i\Pi} & 0 \\
	0 & 1
	\end{pmatrix} =
	\begin{pmatrix}
	e^{2i\Pi} & 0 \\
	0 & 1
	\end{pmatrix} =
	\begin{pmatrix}
	1 & 0 \\
	0 & 1
	\end{pmatrix}
	$$
	
	$\Rightarrow$ порядок группы Г$_{2}$ равен 2.
	
	Рассмотрим ещё одну произвольную конечную группу Г$_{3}$, $\{e^{i\frac{2\Pi}{3}},e^{i\frac{4\Pi}{3}},1\}$.
	
	$$
	\begin{pmatrix} 
	e^{i\frac{2\Pi}{3}} & 0 & 0 \\
	0 & 1 & 0 \\
	0 & 0 & e^{i\frac{4\Pi}{3}}		
	\end{pmatrix}
	$$
	
	Результатом перемножения уже трёх матриц является матрица:
	
	$$
	\begin{pmatrix} 
	e^{i\frac{2\Pi}{3}} & 0 & 0 \\
	0 & 1 & 0 \\
	0 & 0 & e^{i\frac{4\Pi}{3}}		
	\end{pmatrix} \cdot
	\begin{pmatrix} 
	e^{i\frac{2\Pi}{3}} & 0 & 0 \\
	0 & 1 & 0 \\
	0 & 0 & e^{i\frac{4\Pi}{3}}		
	\end{pmatrix} \cdot
		\begin{pmatrix} 
	e^{i\frac{2\Pi}{3}} & 0 & 0 \\
	0 & 1 & 0 \\
	0 & 0 & e^{i\frac{4\Pi}{3}}		
	\end{pmatrix} = 
		\begin{pmatrix} 
	1 & 0 & 0 \\
	0 & 1 & 0 \\
	0 & 0 & 1		
	\end{pmatrix}
	$$
	
	Если группа будет увеличиваться, то будет расти и её порядок $\Rightarrow$ можем сказать, что порядок группы Г$_{n}$ равен $n$. 
	
	\subsubsection{Задание №10(д)}
	\underline{Решение:} Рассмотрим мультипликативную группу классов вычетов $\mathbb{Z}^*_{16}$
	
	$\mathbb{Z}_{16}=$\{1,2,3,4,5,...,15\};
	
	$\mathbb{Z}^*_{16}=$\{1,3,5,7,9,11,13,15\};
	
	Составим таблицу Кэли:
	
	\begin{longtable}{c|cccccccc}
		*&1&3&5&7&9&11&13&15\\\hline
		1&1&3&5&7&9&11&13&15\\
		3&3&9&15&5&11&1&7&13\\
		5&5&15&9&3&13&7&1&11\\
		7&7&5&3&1&15&13&11&9\\
		9&9&11&13&15&1&3&5&7\\
		11&11&1&7&13&3&9&15&5\\
		13&13&7&1&11&5&15&9&3\\
		15&15&13&11&9&7&5&3&1
	\end{longtable}
	
	Тогда найдём порядки для всех элементов:
	
	$ord(1)=1$
	
	$ord(3)=4$
	
	$ord(5)=4$
	
	$ord(7)=2$
	
	$ord(9)=2$
	
	$ord(11)=4$
	
	$ord(13)=4$
	
	$ord(15)=2$
	
	$exp(\mathbb{Z}^*_{16})=[1,4,4,2,2,4,4,2]=4$
	
	\subsubsection{Задание №10(и)}
	\underline{Решение:} Рассмотрим все случаи цикловых структур в группе $S_{n}$:
	
	$(\cdot)(\cdot)(\cdot)(\cdot)(\cdot)$ $\Rightarrow$ $ordS_{5}=1$
	
	$(\cdot$ $\cdot)(\cdot)(\cdot)(\cdot)$ $\Rightarrow$ $ordS_{5}=2$
	
	$(\cdot$ $\cdot$ $\cdot)(\cdot)(\cdot)$ $\Rightarrow$ $ordS_{5}=3$
	
	$(\cdot$ $\cdot)(\cdot$ $\cdot$ $\cdot)$ $\Rightarrow$ $ordS_{5}=6$
	
	$(\cdot$ $\cdot$ $\cdot$ $\cdot)(\cdot)$ $\Rightarrow$ $ordS_{5}=4$
	
	$(\cdot$ $\cdot$ $\cdot$ $\cdot$ $\cdot)$ $\Rightarrow$ $ordS_{5}=5$	
	
	$exp(S_{n})=n!=720$	
	
	\subsubsection{Задание №38}
	$\sqsupset H_{1},H_{2}$ - подгруппы в группе $G$, причём $H_{1}\in H_{2}$. Доказать, что если $|H_{2}:H_1|=n$ и $|G:H_{2}|=m$, то $|G:H_{1}|=mn$.
	
	\underline{Доказательство:}
	
	$\sqsupset$ для конечной группы справедливо: 
	
	По т. Лагранжа для конечных групп, а также её подгрупп:
	
	$|H_{2}:H_{1}|=\frac{|H_{2}|}{|H_{1}|}$
	
	$|G|=m|H_{2}|,|H_{1}|=\frac{H_{2}}{n}$ $\Rightarrow$ $\frac{G}{H_{1}}=\frac{m|H_{2}|n}{|H_{2}|}=mn$.
	
	В то же время рассмотрим бесконечные группы, которые мы можем разбить на бесконечное число конечный а также идентичных подмножеств, для которых справедливо:
	
	Если для $m$ элементов из $G$ $\exists$ 1 элемент $H_{2}$, то и для $km$ элементов $G$ $\exists$ $k$ элементов из $H_{2}$.
	
	\newpage
	
	\subsection{Строение групп}
	\subsubsection{Задание №8}
	$\sqsupset$ $G$ - произвольная группа. Доказать, что равенство 
	
	$|HK|=\frac{|H|\cdot|K|}{|H\cap K|}$
	
	справедливо для всех конечных подгрупп $H,K$ $<$ $G$.
	
	\underline{Доказательство:}
	
	В силу того, что $\forall x$ $\in$ $K:$ $xK$ $=$ $K$:
	
	$\forall x$ $\in$ $(H\cap K), h\in H:h=const\Leftrightarrow (hx)K=const=hK\Rightarrow$
	
	$\Rightarrow \{hK|h\in H\}\cong H/(H\cap K)|\Rightarrow$
	
	$\Rightarrow \big|\{hK|h\in H\}\big|=\big|H/(H\cap K)\big|$
	
	Рассмотрим элементы множества левых классов:
	
	$\forall h \in H:|hK|=const=K\Rightarrow$ 
	
	$\Rightarrow$ Основываясь на т. Лагранжа можно сказать, что:
	
	$|HK|=|K|\cdot\big|H/(H\cap K)\big|=\frac{|H|\cdot|K|}{|H\cap K|}$, ч.т.д.
	
	\subsubsection{Задание №21}
	Доказать, что силовская $p-$подгруппа группы $G$ единственная т. и т. т., когда она нормальна с $G$.
	
	\underline{Доказательство:}
	
	$\sqsupset$ порядок группы $G$ имеет вид: $|G|=p^{n}s,$где $(p,s)=1$.
	
	Опираясь на 1 т. Силова можно сказать, что в группе $G$ $\exists$ подгруппа порядка $p^{n}$ и по условию она единственна. Поскольку по 2 т. Силова все силовские $p-$подгруппы сопряжены, то верно следующее:
	
	$g^{-1}H_{p}g=H_{p}$.
	
	$\Rightarrow$ выполняется определение нормальной подгруппы. 
	
	Обратно:
	
	$\sqsupset$ $\exists$ 2 силовские $p-$подгруппы $H_{p_{1}}$ и $H_{p_{2}}$, такие, что $H_{p_{1}}$,$H_{p_{2}}$ $\in$ $G$. По определению нормальной подгруппы и 2 теоремы Силова будет справедливо следующее:
	
	\begin{equation*}
	\begin{cases}
	g^{-1}H_{p_{1}}g=H_{p_{1}}\\
	g^{-1}H_{p_{2}}g=H_{p_{2}}\Rightarrow H_{p_{1}}=H_{p_{2}}\Rightarrow H - $единственная$.\\
	g^{-1}H_{p_{1}}g=H_{p_{2}}
	\end{cases}
	\end{equation*} 
	
	\subsubsection{Задание №24(е)}
	Доказать, что любая группа порядка $n$ $\in$ $\mathbb N$ коммутативна.
	
	\underline{Доказательство:} Нам дано число 187, его можно разложить на произведение 11 и 17, тогда найдём количество силовских 11-подгрупп и 17-подгрупп.
	
	$s_{11}|187 \cup s_{11}\equiv1(mod11)\Rightarrow s_{11}=1$
	
	$s_{17}|187 \cup s_{17}\equiv1(mod17)\Rightarrow s_{17}=1$
	
	Силовские подгруппы $H_{11}$ и $H_{17}$ - циклические и имеют тривиальное пересечение из чего можно сделать вывод, что $G=H_{11}\dot{+}H_{17}\cong \mathbb Z_{11} \oplus \mathbb Z_{17}$. $\Rightarrow$ группа порядка 187 абелева.
	
	\newpage
	
	\subsection{Конечные группы подстановок}
	\subsubsection{Задание №9}
	Определить, сколько инверсий образует число $n$, стоящее в нижней строке подстановки степени $n$ на $k-$ом месте.
	
	\underline{Решение:} Рассмотрим каноническую запись подстановки степени 5 и $\sqsupset$ k=4:
	
	\[ \left( 
	\begin{array}{cccccc}
	1&2&3&4&5\\
	4&1&2&5&3
	\end{array} 
	\right) \]
	
	видим, что наше число 5 образует ровно одну инверсию в подстановке 5 степени, тогда (n-k) - формула по которой мы вычисляем число инверсий которые образует число n, расположенное на $k-$ом месте в подстановке степени $n$. 
	
	\subsubsection{Задание №10}
	Показать, что от одной перестановки $(a_{1},...,a_{n})$ к другой перестанов-
	ке $(b_{1},...,b_{n})$ тех же элементов можно перейти путём не более чем $n-1$
	транспозиций.
	
	\underline{Решение:} Возьмём группу $S_{3}$, рассмотрим все её подстановки и найдём число транспозиций благодаря которым мы сможем перейти от одной подстановки к другой:
	
	\[ 1 = \left( 
	\begin{array}{ccc}
	1&2&3\\
	1&2&3
	\end{array} 
	\right)\]
	
	\[ 2 = \left( 
	\begin{array}{ccc}
	1&2&3\\
	2&3&1
	\end{array} 
	\right)\]
	
	\[ 3 = \left( 
	\begin{array}{ccc}
	1&2&3\\
	3&1&2
	\end{array} 
	\right)\]
	
	\[ 4 = \left( 
	\begin{array}{ccc}
	1&2&3\\
	1&3&2
	\end{array} 
	\right)\]
	
	\[ 5 = \left( 
	\begin{array}{ccc}
	1&2&3\\
	3&2&1
	\end{array} 
	\right)\]
	
	\[ 6 = \left( 
	\begin{array}{ccc}
	1&2&3\\
	2&1&3
	\end{array} 
	\right)\]
	
	Из подстановки (1) мы можем перейти в подстановку (2) при помощи транспозиций $(3\,2)$ и $(2\,1)$.
	
	Из подстановки (1) мы можем перейти в подстановку (3) при помощи транспозиций $(3\,1)$ и $(1\,2)$.
	
	Из подстановки (1) мы можем перейти в подстановку (4) при помощи транспозиций $(3\,2)$.
	
	Из подстановки (1) мы можем перейти в подстановку (5) при помощи транспозиций $(3\,1)$.
	
	Из подстановки (1) мы можем перейти в подстановку (6) при помощи транспозиций $(3\,2)$.
	
	Из подстановки (2) мы можем перейти в подстановку (3) при помощи транспозиций $(3\,2),(1\,3)$.
	
	Из подстановки (2) мы можем перейти в подстановку (4) при помощи транспозиций $(2\,1)$.
	
	Из подстановки (2) мы можем перейти в подстановку (5) при помощи транспозиций $(3\,2)$.
	
	Из подстановки (2) мы можем перейти в подстановку (6) при помощи транспозиций $(3\,1)$.
	
	Из подстановки (3) мы можем перейти в подстановку (4) при помощи транспозиций $(3\,1)$.
	
	Из подстановки (3) мы можем перейти в подстановку (5) при помощи транспозиций $(1\,2)$.
	
	Из подстановки (3) мы можем перейти в подстановку (6) при помощи транспозиций $(3\,2)$.
	
	Из подстановки (4) мы можем перейти в подстановку (5) при помощи транспозиций $(3\,1)$ и $(1\,2)$.
	
	Из подстановки (4) мы можем перейти в подстановку (6) при помощи транспозиций $(3\,2)$ и $(1\,3)$.
	
	Из подстановки (5) мы можем перейти в подстановку (6) при помощи транспозиций $(3\,2)$ и $(3\,1)$.
	
	Видим, что максимально возможное число транспозиций $n-1$ т.е. 2 в нашем случае.Раз это свойство верно для $S_{3}$, то верно и для $S_{n}$.
	
		\subsubsection{Задание №16(г)}
	Пусть $A$ $\subset$ $S_{n}$ - некоторое множество транспозиций степени $n$ $\in$ $N$. По свойствам графа Г$_{A}$ описать структуру группы $G$ = $<A>$ и определить, является ли множество А системой образующих или базисом группы $S_{n}$.
	
	\underline{Решение:} $A$=\{(1\,9),(2\,6),(3\,5),(4\,8),(5\,6),(6\,9),(7\,9),(8\,10),(10\,2)\}
	
%	\begin{center}
%		\includegraphics[scale=0.5]{ga}\\
%	\end{center}
	
	Видим, что из каждой точки можно перейти в другую притом только одним способом $\Rightarrow$ наша группа является базисом.
	
	\subsubsection{Задание №16(ж)}
	Пусть $A$ $\subset$ $S_{n}$ - некоторое множество транспозиций степени $n$ $\in$ $N$. По свойствам графа Г$_{A}$ описать структуру группы $G$ = $<A>$ и определить, является ли множество А системой образующих или базисом группы $S_{n}$.
	
	\underline{Решение:} $A$=\{(1\,5),(2\,6),(3\,7),(3\,8),(4\,9),(4\,10),(7\,8),(10\,9)\}
	
%	\begin{center}
%		\includegraphics[scale=0.5]{gb}\\
%	\end{center}
	
	Видим, что мы не можем из любой точки перейти любую в другую $\Rightarrow$ наша группа является системой образующих.
	
	\subsubsection{Задание №21(в)}
	Найти централизатор подстановки $g$ $\in$ $S_{6}$.
	
	\underline{Решение:} Найдём количество всех решений, которые входят в наш нормализатор:
	
	\begin{longtable}{c|rr|rrrr}
		№&1&3&2&5&3&6\\\hline
		1&1&3&5&4&6&2\\
		2&1&3&4&6&2&5\\
		3&1&3&6&2&5&4\\
		4&1&3&2&5&4&6\\
		5&3&1&5&4&6&2\\
		6&3&1&4&6&2&5\\
		7&3&1&6&2&5&4\\
		8&3&1&2&5&4&6
	\end{longtable}
	
	Переведём все наши подстановки в цикловую структуру и запишем их в множество нормализатора:
	
	$G=\{\varepsilon,(2\,5\,4\,6),(4\,2)(6\,5),(2\,6\,4\,5),(1\,3)(2\,5\,4\,6),(1\,3)(4\,2)(6\,5),(1\,3)(2\,6\,4\,5),$
	
	$(1\,3),(2\,5\,4\,6),(4\,2)(6\,5),(2\,6\,4\,5),(1\,3)(2\,5\,4\,6),(1\,3)(4\,2)(6\,5),(1\,3)(2\,6\,4\,5),$
	
	$(1\,3)\}$
	
	\subsubsection{Задание №39(а)}
	Определить, является ли группа $G$ $<$ $S_{n}$ транзитивной.
	\underline{Решение:} Рассмотрим группу $G=<(1\,2\,3)(4\,5\,6),(1\,3\,4\,6)>$
	
	Найдём орбиты элементов:
	\begin{longtable}{l|c}
		$G(1)=(1\,2\,3);$&\\
		$G(2)=(2\,3);$&\\
		$G(3)=(3\,1\,4);$&$\Rightarrow$ т.к. орбиты не совпадают, то\\
		$G(4)=(4\,5\,6);$&$G$ не транзитивна;\\
		$G(5)=(5\,6);$&\\
		$G(6)=(6\,4\,1);$&
	\end{longtable}
	
	%Вторая глава
	\newpage
	
	\section{Индивидуальная часть}
	\paragraph*{Задание 1.} Пусть перестановки элементов множества $\overline{1,n}$ задают нижние строки канонических записей подстановок $g,h\in \mathbf{S}_n$.
	\begin{enumerate}[label=\asbuk{enumi})]
		\item Записать подстановки $g$ и $h$ в каноническом виде. Выписать элементы множеств $\mob g$ и $\mob h$, $\fix g$ и $\fix h$.
		\item Найти подстановки $gh$, $hg$, $g^{-1}$, $h^{-1}$.
		\item Разложить подстановки $g$ и $h$ на независимые циклы, указать их цикловые структуры и найти $\ord g$ и $\ord h$.
		\item Разложить подстановки $g$ и $h$ в произведение транспозиций и определить их чётность.
	\end{enumerate}
	\underline{Решение:} Начнём с пункта А
	
	Выпишем подстановки g и h в каноническом виде:
	
	\[ g = \left( 
	\begin{array}{ccccccccccccccccccc}
	1&2&3&4&5&6&7&8&9&10&11&12&13&14&15&16&17&18&19\\
	13&6&4&16&7&3&10&2&9&11&17&19&5&12&15&14&18&1&8
	\end{array} 
	\right) \]
	
	\[ h = \left( 
	\begin{array}{ccccccccccccccccccc}
	1&2&3&4&5&6&7&8&9&10&11&12&13&14&15&16&17&18&19\\
	16&13&18&14&11&7&10&19&1&9&2&17&12&15&6&5&3&8&4
	\end{array} 
	\right) \]
	
	Для перестановок справедливы утверждения:
	
	Mob g=$\lbrace$1,2,3,4,5,6,7,8,10,11,12,13,14,16,17,18,19$\rbrace$
	
	Fix g=$\lbrace$9,15$\rbrace$
	
	Mob h=$\lbrace$1,2,3,4,5,6,7,8,9,10,11,12,13,14,15,16,17,18,19$\rbrace$
	
	Fix h=$\varnothing$
	
	\vspace{\baselineskip}
	
	Перейдём к пункту Б
	
	\[ g \cdot h = \left( 
	\begin{array}{ccccccccccccccccccc}
	1&2&3&4&5&6&7&8&9&10&11&12&13&14&15&16&17&18&19\\
	12&2&14&5&10&18&9&13&1&2&3&4&11&17&6&15&8&16&19
	\end{array} 
	\right) \]
	
	\[ h \cdot g = \left( 
	\begin{array}{ccccccccccccccccccc}
	1&2&3&4&5&6&7&8&9&10&11&12&13&14&15&16&17&18&19\\
	14&5&1&12&17&10&11&8&13&9&6&18&19&15&3&7&4&2&16
	\end{array} 
	\right) \]
	
	\[ g^{-1} = \left( 
	\begin{array}{ccccccccccccccccccc}
	13&6&4&16&7&3&10&2&9&11&17&19&5&12&15&14&18&1&8\\
	1&2&3&4&5&6&7&8&9&10&11&12&13&14&15&16&17&18&19
	\end{array} 
	\right) \]
	
	\[ h^{-1} = \left( 
	\begin{array}{ccccccccccccccccccc}
	16&13&18&14&11&7&10&19&1&9&2&17&12&15&6&5&3&8&4\\
	1&2&3&4&5&6&7&8&9&10&11&12&13&14&15&16&17&18&19
	\end{array} 
	\right) \]
	
	\vspace{\baselineskip}
	
	Перейдём к пункту В
	
	$g=(1\,13\,5\,7\,10\,11\,17\,18)(2\,6\,3\,4\,16\,14\,12\,19\,8)$
	
	$h=(1\,16\,5\,11\,2\,13\,12\,17\,3\,18\,8\,19\,4\,14\,15\,6\,7\,10\,9)$
	
	$[g]=[8^{1},9^{1},1^{2}]$ $\Rightarrow$ ord g=72
	
	$[h]=[19^{1}]$ $\Rightarrow$ ord h=19
	
	\vspace{\baselineskip}
	
	Перейдём к пункту Г
	
	Разложим наши подстановки на произведение транспозиций:
	
	g=(1\,18)(13\,18)(5\,18)(7\,18)(10\,18)(11\,18)(17\,18)(2\,8)(6\,8)(3\,8)(4\,18)(16\,8)(14\,8)(12\,8)(19\,8)
	
	Видим, что у нас нечётное количество транспозиций $\Rightarrow$ перестановка g нечётная
	
	h=(1\,9)(16\,9)(5\,9)(11\,9)(2\,9)(13\,9)(12\,9)(17\,9)(3\,9)(18\,9)(19\,9)(4\,9)(14\,9)(15\,9)(6\,9)(7\,9)(10\,9)(8\,9) 
	
	Видим, что у нас чётное количество транспозиций $\Rightarrow$ перестановка h чётная
	
	\paragraph*{Задание 2.} Пусть перестановки элементов множества $\overline{1,n}$ задают нижние строки канонических записей подстановок $g,h\in \mathbf{S}_n$.
	\begin{enumerate}[label=\asbuk{enumi})]
		\item Доказать, что подстановки $g$ и $h$ сопряжены в $S_{n}$.
		\item Определить число решений уравнений $x^{-1}gx=h$ и $y^{-1}hy=g$.
		\item Составить таблицы, описывающие множества всех решений каждого из уравнений.
		\item Выписать по два произвольных решения каждого из уравнений и осуществить их проверку.
	\end{enumerate}
	\underline{Решение:} Начнём с пункта А
	
	Выпишем нижние строки подстановок g и h:
	
	g=(1,2,6,3,9,4,5,7,8)
	
	h=(6,2,8,4,9,5,3,7,1)
	
	Разложим подстановки g и h на цикловые структуры:
	
	\[ g = \left( 
	\begin{array}{ccccccccccccccccccc}
	1&2&3&4&5&6&7&8&9\\
	1&2&6&3&9&4&5&7&8
	\end{array} 
	\right)=(7,5,9,8)(3,6,4)\]
	
	Где $[g]=[4^{1},3^{1},1^{2}]$
	
	\[ h = \left( 
	\begin{array}{ccccccccccccccccccc}
	1&2&3&4&5&6&7&8&9\\
	6&2&8&4&1&9&3&7&5
	\end{array} 
	\right)=(1,6,9,5)(3,8,7)\]
	
	Где $[h]=[4^{1},3^{1},1^{2}]$
	
	\vspace{\baselineskip}
	
	Перейдём к пункту Б
	
	Число решений для уравнений $x^{-1}gx=h$ и $y^{-1}hy=g$ одинаково так как в g и h одинаковые цикловые структуры. 
	
	Поскольку цикловые структуры подстановок g и h совпадают, то согласно следствию теоремы о порядке нормализатора, множество всех решений заданного уравнения есть правый смежный класс $N_{S_{n}}(g)\cdot f$, где f - произвольное решение. Тогда справедлива формула:
	$|N_{S_{n}}(g)|=\prod_{i=1}^{r}(k_{i})! \cdot l_{i}^{k^{i}}$, где $k_{i}$ - число циклов заданной длины $l_{i}$.
	
	$\Rightarrow$ число решений равно $(1!\cdot4^{1})\cdot(1!\cdot3^{1})\cdot(2!\cdot1^{2})=24$
	
	\vspace{\baselineskip}
	
	Перейдём к пункту В
	
	Составим таблицу для уравнения $x^{-1}gx=h$ со всеми его решениями:
	
	\vspace{\baselineskip}

	\begin{longtable}{c|rrrr|rrr|r|r}
		№&7&5&9&8&3&6&4&1&2\\\hline
		1&1&6&5&9&3&8&7&4&2\\
		2&1&6&5&9&8&7&3&4&2\\
		3&1&6&5&9&7&3&8&4&2\\
		4&6&5&9&1&3&8&7&4&2\\
		5&6&5&9&1&8&7&3&4&2\\
		6&6&5&9&1&7&3&8&4&2\\
		7&5&9&1&6&3&8&7&4&2\\
		8&5&9&1&6&8&7&3&4&2\\
		9&5&9&1&6&7&3&8&4&2\\
		10&9&1&6&5&3&8&7&4&2\\
		11&9&1&6&5&8&7&3&4&2\\
		12&9&1&6&5&7&3&8&4&2\\
		13&1&6&5&9&3&8&7&2&4\\
		14&1&6&5&9&8&7&3&2&4\\
		15&1&6&5&9&7&3&8&2&4\\
		16&6&5&9&1&3&8&7&2&4\\
		17&6&5&9&1&8&7&3&2&4\\
		18&6&5&9&1&7&3&8&2&4\\
		19&5&9&1&6&3&8&7&2&4\\
		20&5&9&1&6&8&7&3&2&4\\
		21&5&9&1&6&7&3&8&2&4\\
		22&9&1&6&5&3&8&7&2&4\\
		23&9&1&6&5&8&7&3&2&4\\
		24&9&1&6&5&7&3&8&2&4\\
	\end{longtable}
	
	Составим таблицу для уравнения $y^{-1}hy=g$ со всеми его решениями:
	
	\vspace{\baselineskip}

	\begin{longtable}{c|rrrr|rrr|r|r}
		№&1&6&5&9&3&8&7&4&2\\\hline
		1&7&5&9&8&3&6&4&1&2\\
		2&7&5&9&8&6&4&3&1&2\\
		3&7&5&9&8&4&3&6&1&2\\
		4&5&9&8&7&3&6&4&1&2\\
		5&5&9&8&7&6&4&3&1&2\\
		6&5&9&8&7&4&3&6&1&2\\
		7&9&8&7&5&3&6&4&1&2\\
		8&9&8&7&5&6&4&3&1&2\\
		9&9&8&7&5&4&3&6&1&2\\
		10&8&7&5&9&3&6&4&1&2\\
		11&8&7&5&9&6&4&3&1&2\\
		12&8&7&5&9&4&3&6&1&2\\
		13&7&5&9&8&3&6&4&2&1\\
		14&7&5&9&8&6&4&3&2&1\\
		15&7&5&9&8&4&3&6&2&1\\
		16&5&9&8&7&3&6&4&2&1\\
		17&5&9&8&7&6&4&3&2&1\\
		18&5&9&8&7&4&3&6&2&1\\
		19&9&8&7&5&3&6&4&2&1\\
		20&9&8&7&5&6&4&3&2&1\\
		21&9&8&7&5&4&3&6&2&1\\
		22&8&7&5&9&3&6&4&2&1\\
		23&8&7&5&9&6&4&3&2&1\\
		24&8&7&5&9&4&3&6&2&1\\
	\end{longtable}
	
	\vspace{\baselineskip}
	
	Перейдём к пункту Г
	
	Проведём проверку двух произвольных решений (под номерами 1 и 2 в обоих случаях) наших уравнений:
	\[ x = \left( 
	\begin{array}{ccccccccc}
	7&5&9&8&3&6&4&1&2\\
	1&6&5&9&3&8&7&4&2
	\end{array} 
	\right) = (1,4,7)(5,6,8,9) \]
	
	\[ x^{-1} = \left( 
	\begin{array}{ccccccccc}
	1&6&5&9&3&8&7&4&2\\
	7&5&9&8&3&6&4&1&2
	\end{array} 
	\right) = (7,4,1)(9,8,6,5) \]
	
	$x^{-1}gx=(7,4,1)(9,8,6,5)\cdot(7,5,9,8)(3,6,4)\cdot(1,4,7)(5,6,8,9)=$
	
	$=(7,3,8)(1,6,5,9)=h$
	
	\[ x = \left( 
	\begin{array}{ccccccccc}
	7&5&9&8&3&6&4&1&2\\
	1&6&5&9&8&7&3&4&2
	\end{array} 
	\right) = (7,1,4,3,8,9,5,6) \]
	
	\[ x^{-1} = \left( 
	\begin{array}{ccccccccc}
	1&6&5&9&8&7&3&4&2\\
	7&5&9&8&3&6&4&1&2
	\end{array} 
	\right) = (1,7,6,5,9,8,3,4) \]
	
	$x^{-1}gx=(1,7,6,5,9,8,3,4)\cdot(7,5,9,8)(3,6,4)\cdot(7,1,4,3,8,9,5,6)=$
	
	$=(1,6,5,9)(3,8,7)=h$
	
	\[ y = \left( 
	\begin{array}{ccccccccc}
	1&6&5&9&3&8&7&4&2\\
	7&5&9&8&3&6&4&1&2
	\end{array} 
	\right) = (1,7,4)(6,5,9,8) \]
	
	\[ y^{-1} = \left( 
	\begin{array}{ccccccccc}
	7&5&9&8&3&6&4&1&2\\
	1&6&5&9&3&8&7&4&2
	\end{array} 
	\right) = (4,7,1)(8,9,5,6) \]
	
	$y^{-1}hy=(7,1,4)(5,6,8,9)\cdot(1,6,5,9)(3,8,7)\cdot(1,7,5)(6,5,9,8)=$
	
	$=(7,5,9,8)(4,3,6)=g$
	
	\[ y = \left( 
	\begin{array}{ccccccccc}
	1&6&5&9&3&8&7&4&2\\
	7&5&9&8&6&4&3&1&2
	\end{array} 
	\right) = (1,7,3,6,5,9,8,4) \]
	
	\[ y^{-1} = \left( 
	\begin{array}{ccccccccc}
	7&5&9&8&6&4&3&1&2\\
	1&6&5&9&3&8&7&4&2
	\end{array} 
	\right) = (4,8,9,5,6,3,7) \]
	
	$y^{-1}hy=(4,8,9,5,6,3,7)\cdot(1,6,5,9)(3,8,7)\cdot(1,7,3,6,5,9,8,4)=$
	
	$=(7,5,9,8)(4,3,6)=g$
	
	\paragraph*{Задание 3.} Определить какую цикловую структуру и чётность могут иметь подстановки порядка $k$ в группе $S_{n}$. Найти количество подстановок каждого из описанных типов.
	
	\underline{Решение:} Имеем порядок $k=8$ в группе $S_{17}$, т.к. порядок группы это НОК длин всех циклов нашей подстановки, то в любом из представлений данной перестановки будет фигурировать хотя бы один цикл длины 8 и все остальные не больше 8.
	
	Иначе говоря, $ord g=[l_{1}^{k_{1}},l_{2}^{k_{2}},...,l_{r}^{k_{r}}]$ 
	
	Мы можем найти число решений этих уравнений по формуле:
	
	
	$|N_{s_{n}}(g)|=\frac{n!}{\prod_{i=1}^{r}(k_{i})! \cdot l_{i}^{k^{i}}}$
	
	Составим список всех возможных циклов в виде которых может быть представлена наша подстановка:
	
	1) $[8^{2},1^{1}] \Rightarrow\frac{17!}{(2! \cdot 8^{2})(1! \cdot 1^{1})}=2778808032000$
	
	2) $[8^{1},4^{2},1^{1}]\Rightarrow\frac{17!}{(1! \cdot 8^{1})(2! \cdot 4^{2})(1! \cdot 1^{1})}=1389404016000$
	
	3) $[8^{1},2^{4},1^{1}]\Rightarrow\frac{17!}{(1! \cdot 8^{1})(4! \cdot 2^{4})(1! \cdot 1^{1})}=115783668000$
	
	4) $[8^{1},4^{1},2^{2},1^{1}]\Rightarrow\frac{17!}{(1! \cdot 8^{1})(1! \cdot 4^{1})(2! \cdot 2^{2})(1! \cdot 1^{1})}=1389404016000$
	
	5) $[8^{1},4^{1},2^{1},1^{3}]\Rightarrow\frac{17!}{(1! \cdot 8^{1})(1! \cdot 4^{1})(1! \cdot 2^{1})(3! \cdot 1^{3})}=926269344000$
	
	6) $[8^{1},4^{1},1^{5}]\Rightarrow\frac{17!}{(1! \cdot 8^{1})(1! \cdot 4^{1})(5! \cdot 1^{5})}=92626934400$
	
	7) $[8^{1}.2^{2},1^{5}]\Rightarrow\frac{17!}{(1! \cdot 8^{1})(2! \cdot 2^{2})(5! \cdot 1^{5})}=46313467200$
	
	8) $[8^{1},2^{1},1^{7}]\Rightarrow\frac{17!}{(1! \cdot 8^{1})(1! \cdot 2^{1})(7! \cdot 1^{7})}=4410806400$
	
	9) $[8^{1},1^{9}]\Rightarrow\frac{17!}{(1! \cdot 8^{1})(9! \cdot 1^{9})}=122522400$
	
	\paragraph*{Задание 4.} Доказать, что отбражение$\varphi:G\rightarrow G$ является гомоморфизмом групп. Найти его образ и ядро. Вычислить порядок факторгруппы $G/Ker\varphi$и определить, какой группе она изоморфна.
	
	\underline{Решение:} Докажем, что отображение $\varphi:G \rightarrow G$ - гомоморфизм групп.
	
	$G=\mathbb {Z}_{14}\oplus\mathbb {Z}_{18}\oplus\mathbb {Z}_{25}$; отображение $\varphi:(g_{1},g_{2},g_{3})\mapsto(9g_{1},16g_{2},15g_{3})$
	
	$\varphi \big((a,b,c)+(d,e,f)\big)=\varphi(a+d,b+e,c+f)=\big(9(a+d),16(b+e),15(c+f)\big)$
	
	$\varphi \big((a,b,c)\big)+\varphi\big((d,e,f)\big)=(9a,16b,15c)+(9d,16e,15f)=\big(9(a+d),16(b+e),15(c+f)\big)$
	
	$\Rightarrow\varphi:G \rightarrow G$ - гомоморфизм.
	
	Порядок группы $G:|G|=14\cdot18\cdot25=6300$
	
	$Im\varphi=\{(9a,16b,15c)\big|a\in\mathbb {Z}_{14},b\in\mathbb {Z}_{18},c\in\mathbb {Z}_{25}\}$
	
	$\mathbb {Z}_{14}=\{\textbf{0},1,2,3,4,5,6,7,8,9,10,11,12,13\}$
	
	$9\mathbb {Z}_{14}=\{0,9,4,13,8,3,12,7,2,11,6,1,10,5\}$
	
	$\mathbb {Z}_{18}=\{\textbf{0},1,2,3,4,5,6,7,8,\textbf{9},10,11,12,13,14,15,16,17\}$
	
	$16\mathbb {Z}_{18}=\{0,16,14,12,10,8,6,4,2\}$
	
	$\mathbb {Z}_{25}=\{\textbf{0},1,2,3,4,\textbf{5},6,7,8,9,\textbf{10},11,12,13,14,$
	$\textbf{15},16,17,18,19,\textbf{20},21,$
	
	$22,23,24\}$
	
	$15\mathbb {Z}_{25}=\{0,15,5,20,10\}$
	
	Ker$\varphi$ = \{(0,0,0)\,(0,9,0)\,(0,0,5)\,(0,9,5)\,(0,0,10)\,(0,9,10)\,(0,0,15)\,
	(0,9,15)\,
	
	(0,0,20)\,(0,9,20)\} $\Rightarrow $всего 10 сочетаний.
	
	$|G/Ker\varphi|=\frac{|G|}{|Ker\varphi|}=\frac{|\mathbb {Z}_{14}\cdot\mathbb {Z}_{18}\cdot\mathbb {Z}_{25}|}{10}=\frac{6300}{10}=630$
	
	По теореме о гомоморфизме групп, $|G/Ker\varphi|\cong Im\varphi =9\mathbb {Z}_{14}\oplus$
	
	$\oplus6\mathbb {Z}_{18}\oplus15\mathbb {Z}_{25};$
	
	$9\mathbb {Z}_{14}=\mathbb {Z}_{14};$
	
	$16\mathbb {Z}_{18}=2\mathbb{Z}_{ 18}\cong\mathbb{Z}_{9};$
	
	$15\mathbb {Z}_{25}=3\mathbb {Z}_{25}\cong\mathbb {Z}_{5};$
	
	$\Rightarrow|G/Ker\varphi|\cong Im\varphi =\mathbb {Z}_{14}\oplus\mathbb {Z}_{9}\oplus\mathbb {Z}_{5}$
	
	\newpage
	
	\paragraph*{Задание 5.} Пусть $G$ и $H$– конечные абелевы группы.
	\begin{enumerate}[label=\asbuk{enumi})]
		\item Вычислить порядки групп $G$ и $H$.
		\item Выписать канонические разложения групп $G$ и $H$.
		\item Найти $typ$ $G$ и $typ$ $H$ и определить, изоморфны ли группы $G$ и $H$.
	\end{enumerate}
	\underline{Решение:} $G=\mathbb {Z}_{51}\oplus\mathbb {Z}_{15}\oplus\mathbb {Z}_{4}\oplus\mathbb {Z}_{187}$
	
	$ord G = 572220$
	
	$H=\mathbb {Z}_{45}\oplus\mathbb {Z}_{187}\oplus\mathbb {Z}_{17}\oplus\mathbb {Z}_{4}$
	
	$ord H = 572220$
	
	Для G справедливо следующее:
	
	$51=17\cdot3;15=3\cdot5;187=17\cdot11$
	
	Для H справедливо следующее:
	
	$45=9\cdot5;187=17\cdot11;$
	
	Канонические записи G и H:

	\vspace{\baselineskip}
	
	$\mathbb {Z}_{4}=<1>$
	
	$\mathbb {Z}_{15}\cong \mathbb {Z}_{3}\oplus\mathbb {Z}_{5}=<5>\dot{+}<3>$
	
	$\mathbb {Z}_{51}\cong\mathbb {Z}_{17}\oplus\mathbb {Z}_{3}=<3>\dot{+}<17>$
	
	$\mathbb {Z}_{187}\cong\mathbb {Z}_{17}\oplus\mathbb {Z}_{11}=<11>\dot{+}<17>$
	
	\vspace{\baselineskip}
	
	$G = <(17,0,0,0)>\dot{+}<(3,0,0,0)>\dot{+}<(0,3,0,0)>\dot{+}$ 
	
	$\dot{+}<(0,5,0,0)>\dot{+}<(0,0,1,0)>\dot{+}<(0,0,0,17)>\dot{+}$
	
	$\dot{+}<(0,0,0,11)> \cong \mathbb {Z}_{17}\oplus\mathbb {Z}_{3}\oplus\mathbb {Z}_{3}\oplus
	\mathbb {Z}_{5}\oplus\mathbb {Z}_{4}\oplus\mathbb {Z}_{17}\oplus\mathbb {Z}_{11}$  
	
	\vspace{\baselineskip}
	
	$\mathbb {Z}_{4}=<1>$
	
	$\mathbb {Z}_{17}=<1>$
	
	$\mathbb {Z}_{45}\cong\mathbb {Z}_{9}\oplus\mathbb {Z}_{5}=<5>\dot{+}<9>$
	
	$\mathbb {Z}_{187}\cong\mathbb {Z}_{17}\oplus\mathbb {Z}_{11}=<11>\dot{+}<17>$
	
	\vspace{\baselineskip}
	
	$H =<(9,0,0,0)>\dot{+}<(5,0,0,0)>\dot{+}<(0,17,0,0)>\dot{+}$
	
	$\dot{+}<(0,11,0,0)>\dot{+}<(0,0,1,0)>\dot{+}<(0,0,0,1)> \cong $ 
	
	$\cong \mathbb {Z}_{9}\oplus\mathbb {Z}_{5}\oplus\mathbb {Z}_{11}\oplus\mathbb {Z}_{17}\oplus\mathbb {Z}_{17}\oplus\mathbb {Z}_{4}$
	
	\vspace{\baselineskip}
	
	$typG=[3^{1},3^{1},2^{2},5^{1},11^{1}17^{1},17^{1}]$ 
	
	$typH=[3^{2},2^{2},5^{1},11^{1}17^{1},17^{1}]$ 
	
	$\Rightarrow G\ncong H$
	\newpage
	\paragraph*{Задание 6.} Пусть $G$ $=$ $<g_{1},g_{2}>$ - группа подстановок степени $n$ $\in$ $N$, порождённая элементами $g_{1}, g_{2}$ $\in$  $S_{n}$.
	\begin{enumerate}[label=\asbuk{enumi})]
		\item Определить, является ли группа $G$ абелевой, и выписать все её эле-менты.
		\item Выписать орбиты и стабилизаторы для каждой из точек a $\in$ $\overline{1,n}$ в группе $G$.
		\item Определить, является ли группа $G$ транзитивной ($k-$транзитивной)или регулярной ($k-$регулярной).
		\item Определить, является ли группа $G$ примитивной или импримитивной.
	\end{enumerate}
	\underline{Решение:} $G=<g_{1},g_{2}>, g_{1}=(1\,4\,6\,2\,5\,3), g_{2}=(1\,2)(3\,5)(4\,6)$
	
	Начнём с пункта А
	
	Найдём все элементы группы G:
	
	По образующей $g_{1}$:
	
	\[ g_{1}\cdot g_{1} = \left( 
	\begin{array}{cccccc}
	1&2&3&4&5&6\\
	6&3&4&2&1&5
	\end{array} 
	\right) = g_{3} \]
	
	\[ g_{1}\cdot g_{2} = \left( 
	\begin{array}{cccccc}
	1&2&3&4&5&6\\
	6&3&2&4&5&1
	\end{array} 
	\right) = g_{4} \]
	
	\[ g_{1}\cdot g_{3} = \left( 
	\begin{array}{cccccc}
	1&2&3&4&5&6\\
	2&1&6&5&4&3
	\end{array} 
	\right) = g_{5} \]
	
	\[ g_{1}\cdot g_{4} = \left( 
	\begin{array}{cccccc}
	1&2&3&4&5&6\\
	4&5&6&1&2&3
	\end{array} 
	\right) = g_{6} \]
	
	\[ g_{1}\cdot g_{5} = \left( 
	\begin{array}{cccccc}
	1&2&3&4&5&6\\
	5&4&2&3&6&1
	\end{array} 
	\right) = g_{7} \]
	
	\[ g_{1}\cdot g_{6} = \left( 
	\begin{array}{cccccc}
	1&2&3&4&5&6\\
	1&2&4&3&6&5
	\end{array} 
	\right) = g_{8} \]
	
	\[ g_{1}\cdot g_{7} = \left( 
	\begin{array}{cccccc}
	1&2&3&4&5&6\\
	3&6&5&1&2&4
	\end{array} 
	\right) = g_{9} \]
	
	\[ g_{1}\cdot g_{8} = \left( 
	\begin{array}{cccccc}
	1&2&3&4&5&6\\
	3&6&1&5&4&2
	\end{array} 
	\right) = g_{10} \]
	
	\[ g_{1}\cdot g_{9} = \left( 
	\begin{array}{cccccc}
	1&2&3&4&5&6\\
	1&2&3&4&5&6
	\end{array} 
	\right) = \varepsilon \]
	
	\[ g_{1}\cdot g_{10} = \left( 
	\begin{array}{cccccc}
	1&2&3&4&5&6\\
	5&4&3&2&1&6
	\end{array} 
	\right) = g_{11} \]
	
	\[ g_{1}\cdot g_{11} = \left( 
	\begin{array}{cccccc}
	1&2&3&4&5&6\\
	2&1&5&6&3&4
	\end{array} 
	\right) = g_{2} \]
	
	\[ g_{1}\cdot g_{1} = \left( 
	\begin{array}{cccccc}
	1&2&3&4&5&6\\
	6&3&4&2&1&5
	\end{array} 
	\right) = g_{3} \]
	
	\[ g_{2}\cdot g_{1} = \left( 
	\begin{array}{cccccc}
	1&2&3&4&5&6\\
	5&4&3&2&1&6
	\end{array} 
	\right) = g_{11} \]
	
	\[ g_{3}\cdot g_{1} = \left( 
	\begin{array}{cccccc}
	1&2&3&4&5&6\\
	2&1&6&5&4&3
	\end{array} 
	\right) = g_{5} \]
	
	\[ g_{4}\cdot g_{1} = \left( 
	\begin{array}{cccccc}
	1&2&3&4&5&6\\
	2&1&5&6&3&4
	\end{array} 
	\right) = g_{2} \]
	
	\[ g_{5}\cdot g_{1} = \left( 
	\begin{array}{cccccc}
	1&2&3&4&5&6\\
	5&4&2&3&6&1
	\end{array} 
	\right) = g_{7} \]
	
	\[ g_{6}\cdot g_{1} = \left( 
	\begin{array}{cccccc}
	1&2&3&4&5&6\\
	6&3&2&4&5&1
	\end{array} 
	\right) = g_{4} \]
	
	\[ g_{7}\cdot g_{1} = \left( 
	\begin{array}{cccccc}
	1&2&3&4&5&6\\
	3&6&5&1&2&4
	\end{array} 
	\right) = g_{9} \]
	
	\[ g_{8}\cdot g_{1} = \left( 
	\begin{array}{cccccc}
	1&2&3&4&5&6\\
	4&5&6&1&2&3
	\end{array} 
	\right) = g_{6} \]
	
	\[ g_{9}\cdot g_{1} = \left( 
	\begin{array}{cccccc}
	1&2&3&4&5&6\\
	1&2&3&4&5&6
	\end{array} 
	\right) = \varepsilon \]
	
	\[ g_{10}\cdot g_{1} = \left( 
	\begin{array}{cccccc}
	1&2&3&4&5&6\\
	1&2&4&3&6&5
	\end{array} 
	\right) = g_{8} \]
	
	\[ g_{11}\cdot g_{1} = \left( 
	\begin{array}{cccccc}
	1&2&3&4&5&6\\
	3&6&1&5&4&2
	\end{array} 
	\right) = g_{10} \]
	
	По образующей $g_{2}$:
	
	\[ g_{2}\cdot g_{1} = \left( 
	\begin{array}{cccccc}
	1&2&3&4&5&6\\
	5&4&3&2&1&6
	\end{array} 
	\right) = g_{11} \]
	
	\[ g_{2}\cdot g_{2} = \left( 
	\begin{array}{cccccc}
	1&2&3&4&5&6\\
	1&2&3&4&5&6
	\end{array} 
	\right) = \varepsilon \]
	
	\[ g_{2}\cdot g_{3} = \left( 
	\begin{array}{cccccc}
	1&2&3&4&5&6\\
	3&6&1&5&4&2
	\end{array} 
	\right) = g_{10} \]
	
	\[ g_{2}\cdot g_{4} = \left( 
	\begin{array}{cccccc}
	1&2&3&4&5&6\\
	3&6&5&1&2&4
	\end{array} 
	\right) = g_{9} \]
	
	\[ g_{2}\cdot g_{5} = \left( 
	\begin{array}{cccccc}
	1&2&3&4&5&6\\
	1&2&4&3&6&5
	\end{array} 
	\right) = g_{8} \]
	
	\[ g_{2}\cdot g_{6} = \left( 
	\begin{array}{cccccc}
	1&2&3&4&5&6\\
	5&4&2&3&6&1
	\end{array} 
	\right) = g_{7} \]
	
	\[ g_{2}\cdot g_{7} = \left( 
	\begin{array}{cccccc}
	1&2&3&4&5&6\\
	4&5&6&1&2&3
	\end{array} 
	\right) = g_{6} \]
	
	\[ g_{2}\cdot g_{8} = \left( 
	\begin{array}{cccccc}
	1&2&3&4&5&6\\
	2&1&6&5&4&3
	\end{array} 
	\right) = g_{5} \]
	
	\[ g_{2}\cdot g_{9} = \left( 
	\begin{array}{cccccc}
	1&2&3&4&5&6\\
	6&3&2&4&5&1
	\end{array} 
	\right) = g_{4} \]
	
	\[ g_{2}\cdot g_{10} = \left( 
	\begin{array}{cccccc}
	1&2&3&4&5&6\\
	6&3&4&2&1&5
	\end{array} 
	\right) = g_{3} \]
	
	\[ g_{2}\cdot g_{11} = \left( 
	\begin{array}{cccccc}
	1&2&3&4&5&6\\
	4&5&1&6&3&2
	\end{array} 
	\right) = g_{1} \]
	
	\[ g_{1}\cdot g_{2} = \left( 
	\begin{array}{cccccc}
	1&2&3&4&5&6\\
	6&3&2&4&5&1
	\end{array} 
	\right) = g_{4} \]
	
	\[ g_{2}\cdot g_{2} = \left( 
	\begin{array}{cccccc}
	1&2&3&4&5&6\\
	1&2&3&4&5&6
	\end{array} 
	\right) = \varepsilon \]
	
	\[ g_{3}\cdot g_{2} = \left( 
	\begin{array}{cccccc}
	1&2&3&4&5&6\\
	4&5&6&1&2&3
	\end{array} 
	\right) = g_{6} \]
	
	\[ g_{4}\cdot g_{2} = \left( 
	\begin{array}{cccccc}
	1&2&3&4&5&6\\
	4&5&1&6&3&2
	\end{array} 
	\right) = g_{1} \]
	
	\[ g_{5}\cdot g_{2} = \left( 
	\begin{array}{cccccc}
	1&2&3&4&5&6\\
	1&2&4&3&6&5
	\end{array} 
	\right) = g_{8} \]
	
	\[ g_{6}\cdot g_{2} = \left( 
	\begin{array}{cccccc}
	1&2&3&4&5&6\\
	6&3&4&2&1&5
	\end{array} 
	\right) = g_{3} \]
	
	\[ g_{7}\cdot g_{2} = \left( 
	\begin{array}{cccccc}
	1&2&3&4&5&6\\
	3&6&1&5&4&2
	\end{array} 
	\right) = g_{10} \]
	
	\[ g_{8}\cdot g_{2} = \left( 
	\begin{array}{cccccc}
	1&2&3&4&5&6\\
	2&1&6&5&4&3
	\end{array} 
	\right) = g_{5} \]
	
	\[ g_{9}\cdot g_{2} = \left( 
	\begin{array}{cccccc}
	1&2&3&4&5&6\\
	5&4&3&2&1&6
	\end{array} 
	\right) = g_{11} \]
	
	\[ g_{10}\cdot g_{2} = \left( 
	\begin{array}{cccccc}
	1&2&3&4&5&6\\
	5&4&3&2&1&6
	\end{array} 
	\right) = g_{7} \]
	
	
	\[ g_{11}\cdot g_{2} = \left( 
	\begin{array}{cccccc}
	1&2&3&4&5&6\\
	3&6&5&1&2&4
	\end{array} 
	\right) = g_{9} \]
	
	\vspace{\baselineskip}
	
	Построим неполную таблицу Кэли, заполнив лишь те столбцы и строки, которые соответствуют $g_{}$ и $g_{2}$:
	
	\vspace{\baselineskip}
	
	\begin{longtable}{c|r|r|r|r|r|r|r|r|r|r|r|r}
	$\cdot$&$\varepsilon$&$g_{1}$&$g_{2}$&$g_{3}$&$g_{4}$&$g_{5}$&$g_{6}$&$g_{7}$&$g_{8}$&$g_{9}$&$g_{10}$&$g_{11}$\\\hline
	$\varepsilon$&&$g_{1}$&$g_{2}$&&&&&&&&\\
	$g_{1}$&$g_{1}$&$g_{3}$&$g_{4}$&$g_{5}$&$g_{6}$&$g_{7}$&$g_{8}$&$g_{9}$&$g_{10}$&$\varepsilon$&$g_{11}$&$g_{2}$\\
	$g_{2}$&$g_{2}$&$g_{11}$&$\varepsilon$&$g_{10}$&$g_{9}$&$g_{8}$&$g_{7}$&$g_{6}$&$g_{5}$&$g_{4}$&$g_{3}$&$g_{1}$\\
	$g_{3}$&&$g_{5}$&$g_{6}$&&&&&&&&\\
	$g_{4}$&&$g_{2}$&$g_{1}$&&&&&&&&\\
	$g_{5}$&&$g_{7}$&$g_{8}$&&&&&&&&\\
	$g_{6}$&&$g_{4}$&$g_{3}$&&&&&&&&\\
	$g_{7}$&&$g_{9}$&$g_{10}$&&&&&&&&\\
	$g_{8}$&&$g_{6}$&$g_{5}$&&&&&&&&\\
	$g_{9}$&&$\varepsilon$&$g_{11}$&&&&&&&&\\
	$g_{10}$&&$g_{8}$&$g_{7}$&&&&&&&&\\
	$g_{11}$&&$g_{10}$&$g_{9}$&&&&&&&&
	\end{longtable}

	
	$G=\{\varepsilon,(1\,4\,6\,2\,5\,3),(1\,2)(3\,5)(4\,6),(1\,6\,5)(2\,3\,4),(1\,6)(2\,3),$
	
	$(1\,2)(3\,6)(4\,5),(1\,4)(2\,5)(3\,6),(1\,5\,6)(2\,4\,3),(3\,4)(5\,6),$
	
	$(1\,3\,5\,2\,6\,4),(1\,3)(2\,6)(4\,5),(1\,5)(2\,4)\}$
	
	Группа G не абелева, что следует из таблицы. В качестве примера рассмотрим результаты произведений $g_{1}\cdot g_{2}$ и $g_{2}\cdot g_{1}$.
	
	Переходим к пункту Б
	
	Выпишем орбиты для всех элементов:
	
	$G(1)=\{1,4,2,6,3,5\}=G(2)=G(3)=G(4)=G(5)=G(6)$
	
	Выпишем стабилизаторы для всех элементов:
	
	$St_{G}(1)=\{\varepsilon,(3\,4)(5\,6)\}$
	
	$St_{G}(2)=\{\varepsilon,(3\,4)(5\,6)\}$
	
	$St_{G}(3)=\{\varepsilon,(1\,5)(2\,4)\}$
	
	$St_{G}(4)=\{\varepsilon,(1\,6)(2\,3)\}$
	
	$St_{G}(5)=\{\varepsilon,(1\,6)(2\,3)\}$
	
	$St_{G}(6)=\{\varepsilon,(1\,5)(2\,4)\}$
	
	Перейдём к пункту В
	
	Группа G является транзитивной, т.к. существует ровно одна орбита для всех элементов, но не является регулярной, т.к. $|G|=12; n=6 \Rightarrow |G|\neq n$.
	
	Выпишем орбиты для стабилизаторов:
	
	$St_{G}(1):\{1\},\{2\},\{3,4\},\{5,6\}$
	
	$St_{G}(2):\{1\},\{2\},\{3,4\},\{5,6\}$
	
	$St_{G}(3):\{1,5\},\{2,4\},\{3\},\{6\}$
	
	$St_{G}(4):\{4\},\{5\},\{1,6\},\{2,3\}$
	
	$St_{G}(5):\{4\},\{5\},\{1,6\},\{2,3\}$
	
	$St_{G}(6):\{1,5\},\{2,4\},\{3\},\{6\}$
	
	Так же группа G не является k-транзитивной (при k $\geq$ 2), т.к не выполняется условие: $|G|\nmid n(n-1)$ (и соответственно не является k-регулярной).
	
	Перейдём к пункту Г
	
	Найдём все делители |G|=12: 1,2,3,4,6,12;
	
	$|St_{G}(1)|=...=|St_{G}(6)|=2;$
	
	Нам нужно придерживаться условия максимальности подгрупп:
	
	$St_{G}(a)<H<G$
	
	Тогда рассмотрим циклическую подгруппу $H$, образующими которой являются 2 элемента из стабилизатора:
	
	$H=<(3\,4)(5\,6),(1\,5)(2\,4)>=$
	
	$=\{\varepsilon,(3\,4)(5\,6),(1\,5)(2\,4),(1\,5\,6)(2\,4\,3),(1\,6\,5)(2\,3\,4)\}<G\Rightarrow$

	$\Rightarrow$ т.к. $St_{G}(1),...,St_{G}(6)$ не является максимальной подгруппой (содержится в $H<G$), то $G$ - импримитивна.
	\newpage
	\section{Заключение}
	Группы подстановок – весьма серьезный раздел математики, к которому нужно относиться с полной серьёзностью и постоянно совершенствовать себя в этом направлении, если желаешь не отставать от развития криптографии и новых тенденций информационной безопасности.
	
	К сожалению, не у всех людей найдётся достаточное количество умственных ресурсов для осознания всех криптографических основ. Однако освоение подобного рода материала может значительно упростить понимание основных математических и криптографических моделей. Эти знания смогут помочь в освоении сложнейших шифров и методов дешифрования, найти путь к новейшим методам шифрования, которые смогут стать революционными.
	
	\newpage
	
	\section{Список литературы}
	
	1.Глухов М.М., Елизаров В.П., Нечаев А.А. Алгебра: Учебник. — 2-ое, испр.
	
	2.и доп. изд. — СПб : Издательство «Лань», 2020 — 608 с.
	
	3.Куликов Л.Я. Алгебра и теория чисел: Учебное пособие для педагогических
	институтов. — М : Высшая школа, 1979 — 559 с.
	
	4.Применко Э.А. Алгебраические основы криптографии: Учебное пособие. —
	2-ое, испр. изд. — М : ЛЕНАНД, 2018 — 288 с.
	
	5.Кострикин А.И. Введение в алгебру. Ч. I: Основы алгебры. — 3-ое, стереотип.
	изд. — М : МЦНМО, 2018 — 272 с.
	
	6.Куликов Л.Я., Москаленко А.И., Фомин А.А. Сборник задач по алгебре и тео-
	рии чисел: Учеб. пособие для студентов физ.-мат. спец. пед. ин-тов. — М : Про-
	свещение, 1993 — 288 с.
	
	7.Сборник задач по алгебре / Под ред. Кострикина А.И. — 2-ое, стереотип. изд.
	— М : МЦНМО, 2015 — 416 с.
	
	8.Фадеев Д.К., Соминский И.С. Задачи по высшей алгебре. — 17-ое, стереотип.
	изд. — СПб : Издательство «Лань», 2008 — 288 с.
	
	9.Ляпин Е.С., Айзенштат А.Я., Лесохин М.М. Упражнения по теории групп:
	Учебное пособие. — 2-ое, стереотип. изд. — СПб : Издательство «Лань», 2010
	— 272 с.
		
\end{document}
